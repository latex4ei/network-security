
% .:: Laden der LaTeX4EI Formelsammlungsvorlage
\documentclass[fs, footer]{latex4ei}

% Dokumentbeginn
% ======================================================================
\begin{document}


% Aufteilung in Spalten
\begin{multicols*}{4}
\fstitle{Network Security}


\section{Introduction}

	\sectionbox{
	\subsection{Security Trends}
	Network security is an issue since \emph{critical infrastructures} in the open systems with a growing user base (\ra \emph{increasing risk}) are threatend by \emph{organized crime}.
	}
	\sectionbox{
	\subsection{Security Threats}
	\textbf{Asymmetric Threat:} Defenders must protect against all exploits on all systems but attackers can attack only a few.

	\textbf{Attacker Motivation:} Ego, Revenge, Destruction, Criminal intend, Aquisition of resources, Acquistion of sensitive information
	}
	\sectionbox{
	\subsection{Security concepts}
	\textbf{CIA Triad}
	\begin{itemize}
	 	\item Confidentiality (prevention of unauthorized disclosure)
	 	\item Integrity (prevention of unauthorized modification or deletion)
	 	\item Availability (prevention of unauthorized withholding)
	 \end{itemize}

	 Also: Authenticity, Accountability, Non-repudiation (Nichtabstreitbarkeit), Privacy

	 \textbf{Passive attacks:} Confidentiality (Content compromise, Traffic analysis)

	 \textbf{Active attacks:} Availabilty (Denial of service), Integrity and Authenticity (Modification, Replay, Fabrication)

	 \textbf{Secure Channel:} secure = authentic (of the sender) and confidential (no eavesdropping)

	 \textbf{Security on OSI-Layers:} Physical (Link encryption), Network (IPSEC), Transport (SSL), Application (SSH)

	 }


\section{(In)securtity, Risk and the Lifecycle of Vulnerablities}


	\sectionbox{
	\subsection{(In)security Landscape}
	General: complexity is bad (but increases fast)

	Security of a system := Security of its weakest link
	}
	\sectionbox{
	\subsection{Vulnerablity Lifecycle}

	\textbf{Security vulnerability} \\
	a weakness in a system allowing an attacker to violate the confidentiality, integrity, availability of the system or the data and applications it hosts

	disagreement on what is a vulnerability possible (it's a a feature not a vulnerability)

	\textbf{CVE}
	Standard names for all publicly known vulnerabilities and security exposures. De facto standard.

	Form: \emph{CVE-Year-AnyDigits}

	\includegraphics[width=\columnwidth]{img/VUL-LC.png}


	\textbf{Numbers:}\\
	50 \% of vulnerabilities known to insiders 30 or more days before disclosure (less-than-zero-day)

	At disclosure:  $\approx 50 \%$ unpatched

	A month after disclosure: $\approx 50 \%$ unpatched \\

	\textbf{Zero-day} Date when the vulnerability becomes known by the public

	\textbf{Zero-day-exploit} Attack that exploits a previously unknown vulnerability
	}
	\sectionbox{
	\subsection{Dynamics of (In) Security}
	\begin{itemize}
		\item extremely high dynamics around the disclosure
		\item exploit availability stays higher than the patch availability
		\item insiders: may know undisclosed vulnerabilites
	\end{itemize}

	\textbf{Gap of insecurity:} Difference between the exploit and patch (the bad a consistently faster than the good)

	}

	\sectionbox{
	\subsection{Risk}
	There is such thing as absolute security

	Trade-Off between security and financial, social, functional, \ldots

	\subsubsection{Risk analysis}
	\begin{itemize}
		\item What assets are we trying to protect?
		\item what are the risk to those assets?
		\item How well does the security solution mitigate those risks?
		\item What other risks does the security solution cause?
		\item What costs and trade-offs does the security solution impose?
	\end{itemize}
	\ra is the trade-off worth if?

	\subsubsection{Risk management}
	Security is relative \Ra manage risks

	Options: avoid, decrease, transfer, accept risk

	Business sense: risk $<$ opportunity
	}


\section{Identity and Authentication}

	\subsection{Identity and Identity Theft}

	\textbf{Identity:} An identity specifies a principal (a unique entity)

	e.g. Individuals, Physical objects, Logical objects, Groups

	\textbf{Identitfy theft:} Identity theft is a crime in which imposters obtain key pieces of personally identifying information and use them for their own personal gain or to do harm

	Variants: Financial, Criminal, Identity cloning, Business

	(12.6 million US-citizens in 2012 with 4.6 billion total damage)

	\subsection{Authentication}

	\textbf{Authentication:} Authentication is the process of verifying an identity claim of an entity. It binds the principal to an identity

	\textbf{Criteria:}
	\begin{itemize}
		\item Something an entity knows (e.g. password, PIN)
		\item Something an entity has (e.g. key, card)
		\item Something an entity is (e.g. biometric characteristic)
		\item rarely used: location, ability
	\end{itemize}

	\textbf{Weak authentication:} checking only one authentication criteria

	\textbf{Strong authentication:} checking two or more authentication criteria

	\subsection{Authentication Protocols}
	\subsubsection{OpenID}

	Standard for decentralised user authentication:
	use existing account to sign in to multiple websites without creating a new password

	\includegraphics[width=\columnwidth]{img/OpenID.png}

	\subsubsection{OAuth}
	A web user (resource owner) grants a
	printing service (client) access to her protected photos stored at a photo sharing service (resource server), without sharing her username and password with the printing service.
	Instead, she authenticates directly with a server trusted by the photo sharing service (authorization server) which issues the printing service delegation-specific credentials (access token).

	\includegraphics[width=\columnwidth]{img/OAuth.png}

	(A) Client requests authorisation from the resource owner or authorization server. \\
	(B) Client receives an authorisation grant by the resource owner. Authorization grant type depends on the method used by the client and supported by the authorisation server to obtain it. \\
	(C) Client requests an access token by authenticating with the authorization server using its client credentials (prearranged between the client and authorization server) and presenting the authorization grant. \\
	(D) Authorisation server validates  client credentials and the authorization grant, and if valid issues an access token.\\
	(E) Client requests protected resource from the resource server. Authentication by access token.\\
	(F) Resource server validates the access token \ra valid \ra serves request\\

	\subsubsection{3D Secure}
	A protocol used widely to authenticate online card transactions.

	\includegraphics[width=\columnwidth]{img/3DSecure.png}
	After entering payment card details on merchant site:
	\begin{itemize}
	\item 3D Secure pops up a password entry form to a bank customer
	\item customer enters a password and, if it was correct,
	\item customer is returned to the merchant website to complete the
	transaction and
	\item the merchant gets an authorisation code to submit to his bank
	\end{itemize}

	\textbf{Problems:} Pop-Up blockers, Activation during shopping, SLL verification not visible, liability shift, weak bank authentication, no password reset procedure, privacy issues\\
	\subsubsection{802.1x}
	Client-server based access control protocol that restricts unauthorised devices from connecting to a (W)LAN through publicly accessible ports

	\includegraphics[width=\columnwidth]{img/8021x.png}

	\textbf{EAP} (Extensible Authentication Protocol): Authentication framework supporting multiple methods

	\textbf{Identity and Security}: Authentication (Who?) , Authorisation (What?) , Access Control, Policy enforcement

	\textbf{Benefits:} Standard based technology, control at link layer, interoperating wifi and wired, centralised user administration

	\textbf{Drawbacks:} Authenticator authentication, Man-in-the-middle, session hijacking


	% Identity
	% Authentication (Def, Trusted 3P, Auth-Prot)
	% Pseudonyms
	% Anonymization (Onion Routing, Attacks, no perfect annonymity)
	\subsection{Anonymization}


\section{Firewalls, IDS and NAT Traversal}

\section{The Domain Name System Security}

\section{Availability and Denial of Service}

\section{Secure Channels: Principles, VPN, SSH}

\section{TLS}

\section{Crypto-Refresher}

\section{Web Application Security: Session State}

\section{Web Application Security: SQL Injection}

\section{Cross-Site Scripting (XSS)}


\section{Malware}

\section{Botnets + Malware Development and Demo}

\section{Security Ecosystem and Detection Failures}

\section{E-Mail Spam}
\subsection{Define Spam}
\textbf{UCE}: Unsolicited Commercial Email

\textbf{UBE}: Unsolicited Bulk Email (not nescessaryliy commercial)
% Ende der Spalten
\end{multicols*}

% Dokumentende
% ======================================================================
\end{document}

% ToDos:
